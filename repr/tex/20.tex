


\h{20个练习}


\hh{init missing-semester-cn}

\begin{lstlisting}
mkdir missing-semester-cn
cd missing-semester-cn
git init -b main
\end{lstlisting}


\hh{设置remote}

\shCode{git remote add origin https://github.com/missing-semester-cn/missing-semester-cn.github.io.git}



\hh{下载并使用github cli(gh)}

项目仓库: \url{https://github.com/cli/cli}


Windows下解压其bin/gh.exe到PATH路径下即可

\hhh{登录}


\shCode{gh auth login}
按提示登录即可

按个人常用认证选项:

\shCode{gh auth login --git-protocol=https --web}


\hh{fork 仓库}

\shCode{gh repo fork https://github.com/missing-semester-cn/missing-semester-cn.github.io.git}


\hh{fork之后设置为fork的仓库}



\begin{lstlisting}
  git remote add upstream https://github.com/missing-semester-cn/missing-semester-cn.github.io.git
  git remote set-url git@github.com:lit-notes/missing-semester-cn.github.io.git
\end{lstlisting}


\hh{查看此时remote}

\shCode{git remote -v}


\begin{verbatim}
origin https://github.com/lit-notes/missing-semester-cn.github.io.git (fetch)
origin https://github.com/lit-notes/missing-semester-cn.github.io.git (push)
upstream https://github.com/missing-semester-cn/missing-semester-cn.github.io.git (fetch)
upstream https://github.com/missing-semester-cn/missing-semester-cn.github.io.git (push)
\end{verbatim}


\hh{pull}

\shCode{git pull}

\begin{verbatim}
remote: Enumerating objects: 133, done.
remote: Counting objects: 100% (124/124), done.
remote: Compressing objects: 100% (47/47), done.
remote: Total 89 (delta 60), reused 66 (delta 42), pack-reused 0 (from 0)
Unpacking objects: 100% (89/89), 30.93 KiB | 52.00 KiB/s, done.
\end{verbatim}


\hh{增加submodule}

\begin{lstlisting}
    cd ..
    git submodule add https://github.com/missing-semester-cn/missing-semester-cn.github.io.git ./missing-semester-cn.github.io
\end{lstlisting}

\hh{添加 solution 为 submodule}

\shCode{git submodule add https://github.com/missing-semester-cn/missing-notes-and-solutions missing-notes-and-solutions}

\begin{verbatim}
Cloning into 'xxx/homework/cs_utils/missing-notes-and-solutions'...
remote: Enumerating objects: 330, done.
remote: Counting objects: 100% (41/41), done.
remote: Compressing objects: 100% (25/25), done.
remote: Total 330 (delta 14), reused 30 (delta 10), pack-reused 289 (from 1)
Receiving objects: 100% (330/330), 14.94 MiB | 5.51 MiB/s, done.
Resolving deltas: 100% (122/122), done.
\end{verbatim}


\hh{重置文件内容}

\hhh{修改文件}
\begin{lstlisting}
cd git/missing-semester-cn.github.io
echo something >> _config.yml
\end{lstlisting}

\hhh{恢复}
\shCode{git checkout HEAD -- _config.yml}


\hh{基于当前本地修改新建分支}

\begin{lstlisting}
git stash
git checkout -b temp
git stash apply
\end{lstlisting}


\hh{提交commit}

\begin{lstlisting}
git add .
git commit -m "feat: xxx"
\end{lstlisting}


\hh{pick新分支的修改}

\begin{lstlisting}
nhash=`git log -1 --format=%h'
git checkout -
git cherry-pick $nhash
\end{lstlisting}

\hh{全局设定默认编辑器}

\begin{lstlisting}
git config --global core.editor vim
\end{lstlisting}

\begin{quote}
否则 Linux 下默认有可能是nano

\end{quote}

\hh{新建github仓库}

首先假定 upstream 已经先设置

\begin{lstlisting}
gh repo create --source=. --remote=upstream
\end{lstlisting}

或者

\begin{lstlisting}
gh repo create xxx
\end{lstlisting}


\hh{上传}
\begin{lstlisting}
git push -u upstream main
\end{lstlisting}

"-u, --set-upstream" 为设置默认的fast-forward源

\hh{获得仓库信息}

\begin{lstlisting}
gh repo view
\end{lstlisting}

输出 描述 和 README


\hh{通过branch+merge来新加特性}

\begin{lstlisting}
git branch new-feat
git checkout new-feat
# ... make some change here, then
git commit -m "impr: some feature"
git checkout -
git merge new-feat
\end{lstlisting}

\hh{添加commit, 从文件中读取内容}

\begin{lstlisting}
git commit -m "@file"
\end{lstlisting}

\begin{quotation}
    假设file中已经写好了commit信息
\end{quotation}

\hh{使用mergetool处理冲突}

在出现合并冲突等问题时:

\begin{lstlisting}
git config merge.tool vimdiff
git mergetool
\end{lstlisting}
